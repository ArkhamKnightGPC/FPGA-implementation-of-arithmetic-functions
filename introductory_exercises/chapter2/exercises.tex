\documentclass{article}

\usepackage{tcolorbox}
\usepackage{amsmath}
\usepackage{amssymb}
\usepackage{float}
\usepackage[linesnumbered]{algorithm2e}

\begin{document}

\section*{Architecture of digital circuits}

\begin{tcolorbox}
\textbf{Example (square root function)}\\
Implement a circuit for the following integer square root algorithm:
\begin{algorithm}[H]
    $r \leftarrow 0$\;
    $s \leftarrow 1$\;
    \While{$s \leq x$}{
        $s \leftarrow s + 2*(r+1) + 1$\;
        $r \leftarrow r + 1$\;
    }
    $sqrt \leftarrow r$
\end{algorithm}
\end{tcolorbox}

\begin{tcolorbox}
\textbf{Exercise 1}\\
Let $S_n = \{x \in \mathbb{Z} : 0 \leq x < 2^n\}$ denote the set of all $n$-bit integers.
A \textit{carry-save adder} implements the transformation
\[
\begin{cases}
    (x_1, x_2, x_3) \in S_n^3 \to (y_1, y_2) \in S_n^2 \\
    x_1 + x_2 + x_3 = y_1 + 2 \cdot y_2
\end{cases}
\]
A \textit{7-to-3 counter} uses carry-save adder components to implement the transformation
\[
\begin{cases}
    (x_1, x_2, x_3, x_4, x_5, x_6, x_7) \in S_n^7 \to (y_1, y_2, y_3) \in S_n^3 \\
    x_1 + x_2 + x_3 + x_4 + x_5 + x_6 + x_7 = y_1 + 2 \cdot y_2 + 4 \cdot y_3
\end{cases}
\]
These components are used in multiplication algorithms, where multiple-operand addition scenarios occur. Generate the following VHDL models of a 7-to-3 counter:
\begin{itemize}
    \item a combinational circuit, made up of four carry-save adders;
    \item a data path including one carry-save adder and several registers where computation is performed over 4 cycles
\end{itemize}
\end{tcolorbox}
\end{document}